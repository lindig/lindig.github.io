\documentclass[a4paper,11pt]{article}
\usepackage{booktabs}
\usepackage{hyperref}
\usepackage{url}              
\usepackage[utf8]{inputenc}
\usepackage[T1]{fontenc}
\usepackage{amsmath}
%\usepackage{ebgaramond}

\hypersetup{
  colorlinks = true,
  linkcolor = [rgb]{0.26,0.34,0.48},
  anchorcolor = [rgb]{0.0,0.0,0.8},
  citecolor = [rgb]{0.0,0.0,0.8},
  filecolor = [rgb]{0.0,0.0,0.8},
  urlcolor = [rgb]{0.0,0.0,0.8},
  pdftitle= {Seat Racing (\today)},
  pdfsubject= {Seat Racing in Rowing (\today)},
  pdfauthor= {Christian Lindig -- lindig@gmail.com}
 }
\urlstyle{tt}


% ------------------------------------------------------------------
% macros
% ------------------------------------------------------------------
\def\ms{ms^{-1}}
\let\mc\multicolumn
\setcounter{MaxMatrixCols}{20}

\title{Seat Racing 4+/4-}
\author{Christian Lindig}
\date{\texttt{lindig@gmail.com}\\[2ex] Draft -- \today}
\begin{document}
\maketitle

\noindent Seat racing is a procedure used by rowing coaches to find the
strongest athletes from a squad for a crew boat. While the general
fitness of an athlete can be observed from land training, his or her
ability to ``move the boat'' within a crew is more difficult to
quantify. Seat racing is used to bring objectivity and transparency to
the selection process of finding the best crew for a boat. This paper
takes a fresh look at a method that works along the following
principles:

\begin{enumerate}
  \item Eight rowers are split repeatedly into two crews of four and
  race against each other in two boats under controlled conditions for a
  fixed distance (like 1000m).  
  
  \item After each race rowers between boats are swapped according to a
  fixed plan.  The process continues over a total of 6~races.

  \item Each rower is either rowing always on starboard (right) or port
  (left). This implies that the eight rowers are comprised of four for
  each side. The 4+/4- in the title refers to the class of boat being
  used: 4~rowers in a boat with or without a coxswain.

\item For each race the finishing time for each boat is recorded. \end{enumerate}

The goal of the method is to infer from the recorded finishing times a
ranking of the athletes (per side) in terms of speed contribution.

\section{Example}

We will discuss the method based on the example in
Table~\ref{tab:races} from \cite{purcer:dataset}. It is comprised of
6~races over 1000m. In each race two boats of 4~rowers race against each
other. Rowers are identified as $1$ to $4$ (rowing on starboard) and
$A$ to $D$ (rowing on port). Each crew has two rowers on each side.

What crews are racing each other is fixed and not random. We will see
that this \emph{swapping matrix} has important properties for this
method to work.

\begin{table}[ht]
\centering
\begin{tabular}{rrrr}
  \toprule
 crew & race & boat & time [s] \\ 
  \midrule
 12AB &   1 & 1 & 211.54 \\ 
 34CD &     & 2 & 211.46 \\ [.8ex]
 24AC &   2 & 1 & 199.46 \\ 
 13BD &     & 2 & 204.30 \\ [.8ex]
 23BC &   3 & 1 & 200.97 \\ 
 14AD &     & 2 & 206.29 \\ [.8ex]
 12CD &   4 & 1 & 202.49 \\ 
 34AB &     & 2 & 199.47 \\ [.8ex]
 24BD &   5 & 1 & 205.79 \\ 
 13AC &     & 2 & 205.98 \\ [.8ex]
 23AD &   6 & 1 & 205.78 \\ 
 14BC &     & 2 & 205.81 \\ 
   \bottomrule
\end{tabular}
  \caption{\label{tab:races} Crews of 4~rowers race pairwise in 6~races
  over 1000m with the resulting times in seconds. Rowers are identified
  as $1$ to $4$ and $A$ to $D$.}
\end{table}

\iffalse
Before looking into the details, let us cut straight to one result; there
are actually more but you would not make a mistake by only looking at
this. Table~\ref{tab:basic} sums up for each rower the time they have
spent in races.

\begin{table}[ht]
\centering
\begin{tabular}{l rrrr}
  \toprule
Rower starboard & 1 & 2 & 3 & 4 \\
Total Time [s]  & 1236.41 & 1226.03 & 1227.96 & 1228.28 \\
  \midrule
Rower port      & A & B & C & D \\ 
Total Time [s]  & 1228.52 & 1227.88 & 1226.17 & 1236.11 \\ 
  \bottomrule
\end{tabular}
  \caption{\label{tab:basic} Total time rowers spent in races; the
  strongest rowers on starboard are 2, 3, 4, 1 and on port C, B, A, D.}
\end{table}

\begin{enumerate}
\item The strongest rowers on starboard are 2, 3, 4, 1 in this order
because they spent an increasing amount of time in races.
\item The strongest rowers on port are C, B, A, D in that order for the
same argument.
\end{enumerate}

Maybe that is all you need to know and you don't wonder:

\begin{itemize}
\item Who is the strongest rower overall? Is it rower 2 because he or
she has the shortest time? No.
\item Times between races varied a lot, maybe because of wind. Does this
create unfairness?
\item Do the time differences within a side reflect actual differences
in contribution to boat speed?
\item If I only have one boat, could I just time the races one after
another? No.
\end{itemize}

The rest of this paper will look into why the method works at all and
the answers to the above questions.
\fi

\section{Problem Definition}

Most descriptions of seat racing in the rowing literature talk about
one rower being faster than another by a margin of time. While this is
what it looks like, it is more the effect than the cause. The speed 
of a crew boat is the result of the effort of its crew and time cannot
be attributed to individual crew members. What is happening is that each
rower contributes power to the propulsion of the boat.  It is helpful to
think about individual power and the combined power of the crew. What
seat racing really aims for is to uncover the power contribution of each
rower.

When a boat with 4~crew members travels a distance $d$ in time $t$, its
average speed is $v=d/t$ and the average power $P$ required
depends on a drag factor $k$:

\begin{equation}
\label{eq:p=kv3}
P = k \times v^3
\end{equation}

Drag factor $k$ depends on the boat's shape, crew weight and other
details. However, we are mostly interested in the relative power of
rowers and can live with an approximation of $k = 2.8 \times 4$. This
gives use for each race the average power of the crew. For example, a
time of 211.54s corresponds to 

\begin{equation*}
P=2.8 \times 4 \times (1000/211.54)^3 = 1183.15\,\textrm{watt}.
\end{equation*}

We make the following assumptions:

\begin{enumerate}
\item The power of a crew is the total of the power of its
      members and 
\item each crew member $m$ produces in each
      race the same amount of power $x_m$.
\end{enumerate}

Power $x$ is the number that we are looking for each rower. Power is an
interesting measure because it is known for each rower from land
training, too, and so it offers a comparison.  Restating
Table~\ref{tab:races} in terms of power using Equation~\ref{eq:p=kv3}
leads to Table~\ref{tab:races_power}.

\begin{table}[ht]
\centering
\begin{tabular}{lrrrrrr}
  \toprule
 crew & race & boat & time [s] & power [W] & adj &  adj.~power [W]\\ 
  \midrule
  12AB &   1 & 1& 211.54 & 1183.15 & 1.10 & 1303.02 \\ 
  34CD &     & 2& 211.46 & 1184.50 &      & 1304.49 \\ [.8ex]
  24AC &   2 & 1& 199.46 & 1411.40 & 0.96 & 1350.62 \\ 
  13BD &     & 2& 204.30 & 1313.45 &      & 1256.89 \\ [.8ex]
  23BC &   3 & 1& 200.97 & 1379.83 & 0.98 & 1354.82 \\ 
  14AD &     & 2& 206.29 & 1275.80 &      & 1252.69 \\ [.8ex]
  12CD &   4 & 1& 202.49 & 1348.99 & 0.94 & 1274.37 \\ 
  34AB &     & 2& 199.47 & 1411.19 &      & 1333.14 \\ [.8ex]
  24BD &   5 & 1& 205.79 & 1285.12 & 1.02 & 1305.56 \\ 
  13AC &     & 2& 205.98 & 1281.57 &      & 1301.95 \\ [.8ex]
  23AD &   6 & 1& 205.78 & 1285.31 & 1.01 & 1304.04 \\ 
  14BC &     & 2& 205.81 & 1284.75 &      & 1303.47 \\ 
   \bottomrule
\end{tabular}
\caption{\label{tab:races_power} Race results restated in terms of
implied crew power. Column \emph{power} is the power based on equation
\ref{eq:p=kv3}. This leads to the total power of both crews in a race to
vary across races. \emph{Adjusted power} corrects this such that the
total power in each race is equal to the avarage total power.}
\end{table}

Race~4 is faster than race~1, maybe because of wind or stream
or the distance was slightly shorter than 1000 meters.  The total power
of the rowers in race~4 therefore is seemingly higher than the total
power of the same rowers in race~1 -- which contradicts our assumptions.
We do not know yet how significant this is, as it affects both crews in a
race, but we can try to adjust the power in each race by a factor
\emph{adj} such that the total power in all races is constant and equals
the average total power.  This results in column \emph{adj.~power} in
Table~\ref{tab:races_power}.

The problem of inferring the power contribution of each rower from the
race results can be stated as a system of linear equations in matrix
form:

$$
  \begin{bmatrix}
   1 & 1 & 0 & 0 & 1 & 1 & 0 & 0 \\ 
   0 & 1 & 0 & 1 & 1 & 0 & 1 & 0 \\ 
   0 & 1 & 1 & 0 & 0 & 1 & 1 & 0 \\ 
   1 & 1 & 0 & 0 & 0 & 0 & 1 & 1 \\ 
   0 & 1 & 0 & 1 & 0 & 1 & 0 & 1 \\ 
   0 & 1 & 1 & 0 & 1 & 0 & 0 & 1 \\ 
   0 & 0 & 1 & 1 & 0 & 0 & 1 & 1 \\ 
   1 & 0 & 1 & 0 & 0 & 1 & 0 & 1 \\ 
   1 & 0 & 0 & 1 & 1 & 0 & 0 & 1 \\ 
   0 & 0 & 1 & 1 & 1 & 1 & 0 & 0 \\ 
   1 & 0 & 1 & 0 & 1 & 0 & 1 & 0 \\ 
   1 & 0 & 0 & 1 & 0 & 1 & 1 & 0 \\ 
  \end{bmatrix} 
  \times
  \begin{bmatrix}
  x_1\\
  x_2\\
  x_3\\
  x_4\\
  x_A\\
  x_B\\
  x_C\\
  x_D\\
  \end{bmatrix}
  =
  \begin{bmatrix}
   1303.02 \\ 
   1350.62 \\ 
   1354.82 \\ 
   1274.37 \\ 
   1305.56 \\ 
   1304.04 \\ 
   1304.49 \\ 
   1256.89 \\ 
   1252.69 \\ 
   1333.14 \\ 
   1301.95 \\ 
   1303.47 \\  
  \end{bmatrix}\\
$$

or more succinctly as

\begin{equation}
\label{eq:sx=p}
S x = P
\end{equation}

where $S$ is the swap matrix, $x$ the power of each rower, and $P$ the
observed crew power (for which we used the adjusted power from
Table~\ref{tab:races_power}).  We are now looking for vector $x$, which
assigns power to each rower, such that this matches the observed power.

\section{Inferring Power}

Solving $Sx=P$ for $x$ is not straight forward: $S$ is not square and
therefore has no inverse.  The left inverse $S'$ with $S'S=I$ does not
exist because $\textrm{rank}(S)= 7 < 8$. But $S$ has a unique
generalised inverse $S^+$ which we can use to describe all solutions.

\begin{align}
  S^+ &= \frac{1}{48}
  \begin{bmatrix}
  +7&-5&-5&+7&-5&-5&-5&+7&+7&-5&+7&+7\\
  +7&+7&+7&+7&+7&+7&-5&-5&-5&-5&-5&-5\\
  -5&-5&+7&-5&-5&+7&+7&+7&-5&+7&+7&-5\\
  -5&+7&-5&-5&+7&-5&+7&-5&+7&+7&-5&+7\\
  +7&+7&-5&-5&-5&+7&-5&-5&+7&+7&+7&-5\\
  +7&-5&+7&-5&+7&-5&-5&+7&-5&+7&-5&+7\\
  -5&+7&+7&+7&-5&-5&+7&-5&-5&-5&+7&+7\\
  -5&-5&-5&+7&+7&+7&+7&+7&+7&-5&-5&-5
  \end{bmatrix}
\end{align}


All solutions for rower power $x$ given crew power $P$ in
Equation~\ref{eq:sx=p} are given by

\begin{equation}
  \label{eq:x=sp}
  x = S^+ P + c 
  \begin{bmatrix}
    +1&+1&+1&+1&-1&-1&-1&-1
  \end{bmatrix}^T
\end{equation}

for an arbitrary constant $c$. The fact that an infinite number of
solutions exist may surprise you and call into the question the method.
But it is not as bad as it sounds. First, you could verify that all
solutions in Table~\ref{tab:solutions} indeed respect
Equation~\ref{eq:sx=p}, i.e., the observed race times.

\begin{table}[ht]
\centering
\begin{tabular}{rrrr}
  \toprule
        & \mc3c{Power $x$ [W]} \\
        \cmidrule{2-4}
  rower & $c=0$ & $c=10$ & $c=30$ \\ 
  \midrule
  1 & 293.40 & 303.40 & 323.40 \\ 
  2 & 343.42 & 353.42 & 373.42 \\ 
  3 & 334.14 & 344.14 & 364.14 \\ 
  4 & 332.80 & 342.80 & 362.80 \\ [0.8ex]
  
  A & 331.67 & 321.67 & 301.67 \\ 
  B & 334.53 & 324.53 & 304.53 \\ 
  C & 342.74 & 332.74 & 312.74 \\ 
  D & 294.82 & 284.82 & 264.82 \\ 
   \bottomrule
\end{tabular}
\caption{\label{tab:solutions} Power assignments that
are consistent with race times. Observe that the difference in
power between rowers of one side (1 to 4 and A to D) is constant across all
solutions. Solutions differ only by shifting power between sides.}
\end{table}

\section{Results}

Table~\ref{tab:solutions} provides the answers we are looking for but we
need to be a little careful with the interpretation:

\begin{enumerate}
\item The strongest rowers on starboard are 2, 3, 4, 1 in this order.
\item The strongest rowers on port side are C, B, A, D in this order.
\item We can't tell who is the strongest rower across sides because power
      shifts between sides are not detected by the method. Parameter $c$
      basically models this shift. Across all solutions, the difference
      in power between rowers of one side remains constant and leads to
      an order independent of $c$.
\item With $c=0$ the combined power of rowers of 1 to 4 and A to D is
      equal. This makes it a somewhat more likely to be the true
      scenario than a solution with a large parameter $c$.
\item The swap matrix $S$ and its generalised inverse $S^+$ are fixed
      and independent of the race results. They can be used to implement
      the method as a spreadsheet.
\item The drag factor $k = 2.8 \times 4$ chosen in
      Equation~\ref{eq:p=kv3} could be adjusted without impacting the
      order of rowers but in order to match the observed power on the
      water to power observed in land training.
\item The swap matrix $S$ used here is not the only possible but it
      needs to be chosen carefully for Equation~\ref{eq:x=sp} to be
      valid.
\end{enumerate}

We have used the adjusted power in Table~\ref{tab:races_power} to
compensate for the fact that the total observed raw power between races
varied. This might not be necessary as long as it affects both crews
equally. Computing the power based on raw power and $c=0$ yields an
almost identical result with the largest difference being 1.21\,W for
rower~D and the difference being smaller than 1\,W for most rowers.

\begin{thebibliography}{10}
\bibitem[1]{purcer:dataset} Mike Purcer.  Seat
Racing Analysis. Excel Spreadsheet. Retrieved May 16.  2020.
\url{https://purcerverance.ca/files}
\end{thebibliography}


\end{document}






